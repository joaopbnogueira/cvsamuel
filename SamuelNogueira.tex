%%%%%%%%%%%%%%%%%
% This is an example CV created using altacv.cls (v1.1.3, 30 April 2017) written by
% LianTze Lim (liantze@gmail.com), based on the
% Cv created by BusinessInsider at http://www.businessinsider.my/a-sample-resume-for-marissa-mayer-2016-7/?r=US&IR=T
%
%% It may be distributed and/or modified under the
%% conditions of the LaTeX Project Public License, either version 1.3
%% of this license or (at your option) any later version.
%% The latest version of this license is in
%%    http://www.latex-project.org/lppl.txt
%% and version 1.3 or later is part of all distributions of LaTeX
%% version 2003/12/01 or later.
%%%%%%%%%%%%%%%%

%% If you want to use \orcid or the
%% academicons icons, add "academicons"
%% to the \documentclass options.
%% Then compile with XeLaTeX or LuaLaTeX.
% \documentclass[10pt,a4paper,academicons]{altacv}

%% Use the "normalphoto" option if you want a normal photo instead of cropped to a circle
% \documentclass[10pt,a4paper,normalphoto]{altacv}

\documentclass[10pt,a4paper]{altacv}

%% AltaCV uses the fontawesome and academicon fonts
%% and packages.
%% See texdoc.net/pkg/fontawecome and http://texdoc.net/pkg/academicons for full list of symbols.
%% When using the "academicons" option,
%% Compile with LuaLaTeX for best results. If you
%% want to use XeLaTeX, you may need to install
%% Academicons.ttf in your operating system's font %% folder.


% Change the page layout if you need to
\geometry{left=1cm,right=9cm,marginparwidth=6.8cm,marginparsep=1.2cm,top=1cm,bottom=1cm}

% Change the font if you want to.

% If using pdflatex:
\usepackage[utf8]{inputenc}
\usepackage[T1]{fontenc}
\usepackage[default]{lato}
\usepackage{tabularx}
\usepackage{lmodern,textcomp}
\usepackage[hidelinks]{hyperref}
\usepackage[sfdefault]{roboto}
\usepackage{academicons}
\usepackage{setspace}

% If using xelatex or lualatex:
% \setmainfont{Lato}

% Change the colours if you want to
\definecolor{VividPurple}{HTML}{0E4D92}
\definecolor{SlateGrey}{HTML}{4682B4}
\definecolor{LightGrey}{HTML}{666666}
\colorlet{heading}{VividPurple}
\colorlet{accent}{VividPurple}
\colorlet{emphasis}{SlateGrey}
\colorlet{body}{LightGrey}
\setstretch{1.05}

\renewcommand{\itemmarker}{{\small\textbullet}}
\renewcommand{\ratingmarker}{\faCircle}

\begin{document}
\name{Samuel Nogueira}
\tagline{}
\photo{2.5cm}{foto_square.png}
\personalinfo{%
	\begin{spacing}{1.5}
		\birthday{27/12/1995}
		\location{\href{https://www.google.com/maps/@?api=1\&map\_action=map\&center=39.729449\%2C-8.740353\&zoom=10}{Leiria, Portugal}}		
		\email{\href{mailto:samuel.brites.nogueira@gmail.com}{samuel.brites.nogueira@gmail.com}}  
		\phone{\href{tel:+351910219662}{+351 910 219 662}}
%		\skype{\href{skype:joao.p.nogueira}{joao.p.nogueira}}
%		\linkedin{\href{https://www.linkedin.com/in/joaopbnogueira/}{linkedin.com/in/joaopbnogueira}}
%		\orcid{\href{https://orcid.org/0000-0002-5748-833X}{orcid.org/0000-0002-5748-833X}}
	\end{spacing}
}

%% Make the header extend all the way to the right, if you want.
\begin{fullwidth}
\makecvheader
\end{fullwidth}

%% Provide the file name containing the sidebar contents as an optional parameter to \cvsection.
%% You can always just use \marginpar{...} if you do
%% not need to align the top of the contents to any
%% \cvsection title in the "main" bar.
\cvsection[p1sidebar]{Características Chave}

\smallskip

\begin{tabularx}{\linewidth}{X X}
	• Proactivo & • Bom comunicador  \\
	• Espírito de equipa         & • Vontade de aprender     \\
	• Rigoroso    & • Dedicado
\end{tabularx}

\cvsection{Educação}

\cvtraining{Licenciatura em Engenharia Mecânica}{\href{https://www.isec.pt/PT/estudar/licenciaturas/EngMecanica/}{Instituto Superior de Engenharia de Coimbra (ISEC)}}{2015 -- Jan 2019 (est.)}{13 valores}

\location{Coimbra, Portugal}

\medskip

\begin{itemize}
	\item Estudo de diversas áreas científicas, com ênfase em matemática e física, tendo desenvolvido o \textbf{racicínio lógico e rigoroso};
	\item Desenvolvimento de competências técnicas nos diferentes processos tecnológicos de fabrico, incluindo principalmente sistemas de fabrico assistido por computador, através de desafiantes projectos práticos que potenciam o \textbf{trabalho em equipa, espírito crítico, criatividade, resiliência e capacidade de trabalho}
\end{itemize}

\smallskip


\cvtag{SolidWorks} \cvtag{MastercamCAM} \cvtag{AutoCad} \cvtag{MATLAB} \cvtag{Siemens TiaPortal} \cvtag{Lotus Simulation}

\cvsection{Projectos}

\cvtraining{Despeja Palox Automatizado}{Projeto final de curso}{2017}{18 valores}

\begin{itemize}
	\item Projetado no Software de CAD SolidWorks, tendo sido feito o cálculo estrutural analiticamente e pelo Software através do método dos \textbf{elementos finitos}.
	\item \textbf{Programação de Autómatos} desenvolvido através do Software “Siemens TIAPortal”
	\item \textbf{Órgãos de Máquinas} dimensionados analiticamente
\end{itemize}

\divider

\cvtraining{Saca-Rolhas}{Projeto Laboratórios de Eng. de Produção}{2016}{17 valores}

\begin{itemize}
	\item Projeto com o intuito de solidificar as competências técnicas na área da \textbf{Programação CNC}, usando o Software SolidWorks para projetar e o Software CAD/CAM MasterCAM para gerar o código CNC
	
\end{itemize}

\divider

\cvtraining{Vaso}{Projeto de Fabrico de Moldes}{2018}{14 valores}

\begin{itemize}
	\item Projeto com o intuito de solidificar as competências técnicas na área.
	Molde projetado e dimensionado no Software SolidWorks, recorrendo ao Plastics do mesmo para a simulação de injeção.
\end{itemize}

\smallskip

\cvtag{SolidWorks} \cvtag{MasterCAM} \cvtag{AutoCad} \cvtag{MATLAB} \cvtag{Siemens TiaPortal} \cvtag{Lotus Simulation}


\end{document}
